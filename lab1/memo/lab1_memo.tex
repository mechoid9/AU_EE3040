\documentclass{letter}
\usepackage[top=1in, bottom=1in, left=1in, right=1in]{geometry}
\usepackage{graphicx}
\title{ Title   }
\author{Brian Arnberg}
\date{\today}
\usepackage{setspace}
%\doublespacing
\setlength{\topmargin}{0in}
\setlength{\headheight}{0in}
\setlength{\headsep}{0in}
\setlength{\footskip}{0in}
\begin{document}\label{start}
\pagestyle{empty}
\large
TO: Drs. Victor P. Nelson and John Y. Hung \newline
FROM: ``Uncovered Warriors," Stephen Taylor, and \underline{Brian Arnberg} (Section 3)\newline
DATE: August 26, 2011\newline 
SUBJECT: Project Creation and Debugging\newline

\doublespacing
The primary objective of the first lab session was to familiarize each student 
with the development process for programming the Freescale MC9S12C32 in 
the \emph{CodeWarrior} Development studio. The primary objective was achieved
by having each team program the chip with two pre-written programs.
The secondary objective of the lab was for each student to begin practicing 
and using the communications skills discussed in the lecture. 
The secondary objective was achieved by having each student begin maintaining
his engineering notebook   and by having him write a two-page memo concerning
the lab. 

The two programs used to test the development software were written in assembly and C. Before a team could test the software, though, it had to prepare the 
chip and board for both the download of the program and the testing. 
Setting up the board was done per the explanation in the lab manual. The 
programs were downloaded to the chip via the P\&E USB BDM Multilink module.
Once the board and chip were set up, \emph{CodeWarrior} was set up per the 
instructions in the lab manual. Brian Arnberg's terminal would not correctly 
load the software, so Stephen Taylor's had to be used. This would not be worth
mentioning, if it hadn't taken five minutes for the computer to login. 
After \emph{CodeWarrior} was set up, each team tested the assembly
code given in the manual. It had two errors, one on line 35 (``ldx" was written 
``lx") and one on line 37 (the variable ``Inner" had been typed ``Iner"). Once
the code was corrected, it compiled correctly and downloaded successfully to 
the chip. Once each team was satisfied with the assembly code, it went on to
test the C code, which was without error. Even though the C code was without 
error, Brian had incorrectly copied part of the code, thus there was an 
issue with the compilation. Once the typo that Brian had produced was found, it 
was easily corrected. Once the typo was corrected, the
code correctly compiled and downloaded to the chip. As with the assembly code,
the C code functioned as it was described in the lab manual. Because of the 
work done in the lab, each student now has a better understanding of the 
process used to develop code for the Freescale MC9S12C32, debug the code,
and properly download the code to the chip. Therefore, the primary objective
was achieved. 

Beginning to maintain the engineering notebook was not very difficult. As each
team stepped through the lab manual, each student made notes in his notebook
concerning what they had done. For instance, each student noted which 
lines had errors with it, what the issues were, and how they were fixed. 
Each team might have also made note of any issues they had while they were
setting \emph{CodeWarrior} up. Additional notes, like the amount of time it
took to complete certain steps, might also have been recorded. Once
each student submits his memo, if he has also started writing in his 
notebook, the secondary objective will be complete.  

Both of the programs used did the same thing: They waited for input from a 
push button so that the state of an LED could be toggled. The code accomplished
this by polling the first bit, PE0, until it was 1. When it was one, there was a
delay of about a second, and the state of the LED was toggled by complementing 
the state of the LED bit, PA0. The C code was more understandable, so it is a 
good thing that the rest of the programming done for the semester will
be done in C. 

Code development in \emph{CodeWarrior} was not the only thing that students 
learned about in class. For instance, Brian learned that a Windows 7 machine
cannot read a USB storage device that has been formatted as ext3 and that
he would have to bring a storage device formatted as fat32 or ntfs.
Considering that both of the objectives were met, it would have to be concluded
that this was a successful lab. 
\label{end}\end{document}

